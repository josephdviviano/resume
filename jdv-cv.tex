% Pedigree fools fools.
\documentclass[a4paper,11pt,oneside]{book}
\usepackage[top=0.75in, bottom=0.75in, left=1.25in, right=1.25in]{geometry} % margins
\NeedsTeXFormat{LaTeX2e}
\usepackage{scrextend} % for indenting blocks of text
\usepackage{xcolor}    % for colors
\usepackage{hyperref}
\usepackage{datetime}  % for compile date
\usepackage{multicol}  % column formatting
\usepackage{enumitem}  % for bullets
\usepackage{textcomp}  % for trademark.


\RequirePackage{parskip}

% colour
\definecolor{highlight}{rgb}{0.75,0.1,0.1}
\definecolor{lightgray}{gray}{0.25} % made darker
\colorlet{textcolor}{black}
\colorlet{headercolor}{black}

% font
\RequirePackage{fontspec}
\newfontfamily\bodyfont[]{Brandon Grotesque Regular}
\newfontfamily\thinfont[]{Brandon Grotesque Light}
\newfontfamily\headingfont[]{Brandon Grotesque Bold}
\defaultfontfeatures{Mapping=tex-text}
\setmainfont[Mapping=tex-text, Color=textcolor]{Brandon Grotesque Light}

% formatting
\newcounter{colorCounter}
\def\@sectioncolor#1#2#3{%
  {%
    \color{%
      \ifcase\value{colorCounter}%
        blue\or%
        red\or%
        orange\or%
        green\or%
        purple\or%
        brown\else%
        headercolor\fi%
    } #1#2#3%
  }%
  \stepcounter{colorCounter}%
}

\renewcommand{\section}[1]{
  \par\vspace{\parskip}
  {%
    \LARGE\headingfont\color{headercolor}%
    \@sectioncolor #1%
  }
  \par\vspace{\parskip}
}

\renewcommand{\subsection}[2]{
  \par\vspace{.5\parskip}%
  \Large\headingfont\color{headercolor} #2%
  \par\vspace{.25\parskip}%
}

\pagestyle{empty}

% list environment
\setlength{\tabcolsep}{0pt}
\newenvironment{entrylist}{%
  \begin{tabular*}{\textwidth}{@{\extracolsep{\fill}}ll}
}{%
  \end{tabular*}
}
\renewcommand{\bfseries}{\headingfont\color{headercolor}}
\newcommand{\entry}[4]{%
  \parbox[t]{3cm}{\hfill#1&}\hspace{1cm}\parbox[t]{10.65cm}{%
    \textbf{#2}%
    \hfill%
    {\footnotesize\addfontfeature{Color=lightgray} #3}\\%
    \begin{itemize}[leftmargin=0in]#4\end{itemize}\vspace{\parsep}%
  }}

\renewcommand{\bfseries}{\headingfont\color{headercolor}}
\newcommand{\skills}[2]{%
  \parbox[t]{3cm}{\hfill#1&}\hspace{1cm}\parbox[t]{10.65cm}{%
     \textbf{#2}{\parsep}%
  }\\}

\newcommand\sectionheading{
    \normalsize
    \noindent
    \leftskip=0in
    \textbf
}

%%%%%%%%%%%%%%%%%%%%%%%%%%%%%%%%%%%%%%%%%%%%%%%%%%%%%%%%%%%%%%%%%%%%%%%%%%%%%%%
\begin{document}
\thispagestyle{empty} % remove number on first page

\fboxsep0pt
\noindent
%\colorbox{bg}{
\begin{minipage}{\textwidth}

    \vspace{0.1in}
    \begin{multicols}{2}

        \topskip0pt
        \vspace*{\fill}
        \huge\textbf{JOSEPH D VIVIANO}
        \vspace*{\fill}

        \columnbreak\normalsize
        \hfill www.viviano.ca\

        \hfill joseph@viviano.ca\

        \hfill Mila Quebec AI Instuitute\

        \hfill 6666 St Urbain St, Montr\'eal, QC\

    \end{multicols}
    \vspace{0in}
\end{minipage}
%}
\begin{flushleft}

%%%%%%%%%%%%%%%%%%%%%%%%%%%%%%%%%%%%%%%%%%%%%%%%%%%%%%%%%%%%%%%%%%%%%%%%%%%%%%%%
\sectionheading{EDUCATION} \\\

\entry
{2018--2020}
{MSc. Professional, Computer Science.}
{Universit\'e de Montr\'eal, QC/CA.}
{Internship Supervised by Dr. Yoshua Bengio. \\\
`Methods for controlling and utilizing saliecy maps in medical imaging'.} \\\

\entry
{2011--2013}
{MSc. with Distinction, Biology.}
{York University, ON/CA.}
{Thesis Supervised by Dr. Keith Schneider. \\\
`Tremotopic mapping of the human thalamic reticular nucleus'.\\\
}
\entry
{2005--2009}
{BSc. Hons., Psychology.}
{Queen's University, ON/CA.}
{} \\\

%%%%%%%%%%%%%%%%%%%%%%%%%%%%%%%%%%%%%%%%%%%%%%%%%%%%%%%%%%%%%%%%%%%%%%%%%%%%%%%%
\sectionheading{PUBLICATIONS \& CONFERENCE PAPERS} \\\

* = equal contributions. \\\

\textcolor{highlight}{Viviano JD}, Simpson B, Dutil F, Bengio Y, Cohen JP. Saliency is a possible red herring when diagnosing poor generalization. International Conference on Learning Representations (ICLR), 2021. \\\

Overton DJ, Bhagwat N, \textcolor{highlight}{Viviano JD}, Jacobs GR, Voineskos AN. Identifying psychosis spectrum youth using support vector machines and cerebral blood perfusion as measured by arterial spin labeled fMRI. NeuroImage: Clinical. 27, 102304. \\

Viviano J, Klauer D, Olmos S, \textcolor{highlight}{Viviano JD}. Retrospective comparison of the George Gague\texttrademark registration and the sibilant phoneme registration for constructing OSA oral appliances. CRANIO, 1-9. \\

Lemire-Rodger S, Lam J, \textcolor{highlight}{Viviano JD}, WD Stevens, Spereng RN, Turner GR. 2019. Inhibit, switch, and update: a within-subject fMRI investigation of executive control. Neuropsychologia. 132, 107-134. \\\

Jacobs GR, Ameis SH, Ji JL, \textcolor{highlight}{Viviano JD}, Dickie EW, Wheeler AL, Stojanovski S, Anticevic A, Voineskos AN. 2019. Developmentally divergent sexual dimorphism in the cortico-striatal-thalamic-cortical psychosis risk pathway. Neuropsychopharmacology. 44, 1649-1658. \\\

Hawco C, Buchanan RW, Calarco N, Mulsant BH, \textcolor{highlight}{Viviano JD}, Dickie EW, Argylean M, Gold JM, Iacoboni M, DeRosse P, Foussias G, Malhorta AK, Voineskos AN, for the SPINS group. 2019. Seperable and replicable neural strategies during social brain function with and without severe mental illness. American Journal of Psychiatry. Jan 4 (Online). \\\

\textcolor{highlight}{Viviano JD}, Buchanan RW, Calarco N, Gold J, Foussias G, Bhagwat N, Stefanik L, Hawco C, Malhotra AL, Voineskos AN, for the SPINS group. 2018. Resting-state connectivity biomarkers of cognitive performance and social function in schizophrenia spectrum disorders and healthy controls. Biological Psychiatry. 84(9), 665-674.  \\\

Stojanovski S, Felsky D, \textcolor{highlight}{Viviano JD}, Shahab S, Bangali R, Burton C, Devenyi GA, O'Donnell LJ, Szatmari P, Chakravarty MM, Ameis S, Schachar R, Voineskos AN, Wheeler AL. 2018. Polygenic risk and neural substrates of attention-deficit/hyperactivity disorder symptoms in youth with a history of mild traumatic brain injury. Biological Psychiatry. 85(5), 408-416. \\\

Hawco C, Voineskos AN,  Steeves J, Dickie EW, \textcolor{highlight}{Viviano JD}, Downar J, Blumberger D, Daskalakis ZJ. 2018. Spread of activity following TMS is related to intrinsic resting connectivity to the salience network: a concurrent TMS-fMRI study. Cortex. 108, 160-172. \\\

Bhagwat N, \textcolor{highlight}{Viviano JD}, Voineskos AN, Chakravarty MM. 2018. Modeling and prediction of clinical symptom trajectories in Alzheimer's disease using longitudinal data. 2018. PLoS Computational Biology. 14(9), e1006376. \\\

Chavez S, \textcolor{highlight}{Viviano JD}, Zamyadi M, Kingsley PB, Kochunov P, Strother S, Voineskos AN. 2018. A novel DTI-QA tool: automated metric extraction exploiting the sphericity of an agar filled phantom. Magnetic Resonance Imaging. 46, 28-39. \\\

Hawco C, \textcolor{highlight}{Viviano JD}, Chavez S, Dickie EWE, Calarco N, Kochunov P, Argyelan M, Turner J, Malhortra AK, Buchanan RW, Voineskos AN. 2018. A longitudinal human phantom study of multi-center T1-weighted, DTI, and resting-state fMRI data. Psychiatric Research: Neuroimaging. June 9 (Online).  \\\

Dickie EW, Anticevic A, Smith DE, Coalson TS, Manogaran M, Calarco N, \textcolor{highlight}{Viviano JD}, Glasser MF, Van Essen DC, Voineskos AN. 2018. ciftify: A framework for surface-based analysis of legacy MR acquisitions. Neuroimage. 197, 818-826. \\\

Dickie EW, Ameis SH, \textcolor{highlight}{Viviano JD}, Smith DE, Calarco N, Shahab S, Voineskos AN. 2018. Personalized intrinsic network topography mapping and functional connectvitiy deficits in autism spectrum disorder. Biological Psychiatry. 84(4), 278-286. \\\

\textcolor{highlight}{Viviano JD}, Park MTM, Voineskos AN, Chakravarty MM. 2018. Homology of functional connectivity and structural covariance between the human cerebellum and cortex. Under review. \\\

Kochunov P*, Dickie EW*, \textcolor{highlight}{Viviano JD}*, Turner J, Kingsley PB, Jahanshad N, Thompson P, Ryan M, Fiermans E, Novikov D, Hong EL, Malhotra AK, Buchanan RW, Chavez S, Voineskos AN. 2017. Integration of routine QA data into mega-analysis may improve quality and sensitivity of multi-site diffusion tensor imaging studies. Human Brain Mapping. 39(2), 1015-1023. \\\

Hawco C, Kovacevic N, Malhotra AK, Buchanan RW, \textcolor{highlight}{Viviano JD}, Iacoboni M, McIntosh AR, Voineskos AN. 2017. Neural activity while imitating emotional faces is related to both lower and higher-level social cognitive performance. Scientific Reports. 7, 1244. \\\

Spreng NR, Stevens WD, \textcolor{highlight}{Viviano JD}, Schacter D. 2016. Attenuated anticorrelation between the default and dorsal attention networks with aging: Evidence from task and rest. Neurobiology of Aging. 45, 149-160. \\\

Ameis SH, Lerch JP, Taylor M, Lee W, \textcolor{highlight}{Viviano JD}, Pipitone J, Nazeri A, Croarkin P, Voineskos A, Crosbie J, Brian J, Soreni N, Schachar R, Arnold P, Anagnostou E. 2016. A diffusion tensor imaging study in children with ADHD, autism spectrum disorder, OCD, and matched controls: distinct and non-distinct white matter disruption and dimensional brain-behavior relationships. American Journal of Psychiatry. 173(12):1213-1222. \\\

Wheeler AL, Felsky D, \textcolor{highlight}{Viviano JD}, Stojanovski S, Ameis SH, Szatmari P, Lerch JP, Chakravarty MM, Voineskos AN. 2016. BDNF and sex dependent effects on amygdala-cortical connectivity and depression risk in children and youth. 2017. Cerebral Cortex. 1-11. \\\

\textcolor{highlight}{Viviano JD}, Schneider KA. 2015. Interhemispheric interactions of the human thalamic reticular nucleus. Journal of Neuroscience, 35(5):2026-32. \\\

DeSimone K, \textcolor{highlight}{Viviano JD}, Schneider KA. 2015. Population receptive field estimation reveals new retinotopic maps in human subcortex. Journal of Neuroscience, 35(27):9836-47. \\\

McKetton L, Williams J, \textcolor{highlight}{Viviano JD}, Yücel YH, Gupta N, Schneider KA. 2015. High-resolution structural magnetic resonance imaging of the human subcortex in vivo and postmortem. Journal of Visualized Experiments, 106. \\\

%%%%%%%%%%%%%%%%%%%%%%%%%%%%%%%%%%%%%%%%%%%%%%%%%%%%%%%%%%%%%%%%%%%%%%%%%%%%%%%%
\sectionheading{CONFERENCE TALKS} \\\

Siddiqui S, Saperia S, Da Silva S, Jeffay E, Pipitone J, \textcolor{highlight}{Viviano J}, Zawadzki J, Wong A, Fervaha G, Agid O, Zakzanis K, Remington G, Voineskos A, Foussias G. 2017. Behavioral and neurobiological correlates of attention in schizophrenia in a virtual environment [Abstract]. \textit{Schizophrenia Bulletin, 43, S42}. \\\

\textcolor{highlight}{Viviano JD}, Chavez S, Pipitone JC, Voineskos AN. 2015. Tracking the quality of ongoing acquisitions. \textit{Human Brain Mapping}. \\\

Stevens WDD, Spreng RN, \textcolor{highlight}{Viviano JD}, Schacter DL. 2014. Age-related dedifferentiation of intrinsic functional connectivity both within and between the default and dorsal attention networks. \textit{Society for Neuroscience}. \\\

\textcolor{highlight}{Viviano JD}, Schneider KA. 2013. Imaging the human visual thalamic reticular nucleus. \textit{Society for Neuroscience}. \\\

DeSimone K, \textcolor{highlight}{Viviano JD}, Schneider KA. 2013. Population receptive field estimation in the human subcortical nuclei. \textit{Society for Neuroscience}. \\\

%%%%%%%%%%%%%%%%%%%%%%%%%%%%%%%%%%%%%%%%%%%%%%%%%%%%%%%%%%%%%%%%%%%%%%%%%%%%%%%%
\sectionheading{CONFERENCE POSTERS \& ABSTRACTS} \\\

Hawco C, Dickie EW, \textcolor{highlight}{Viviano J}, Voineskos AN. Clustering multiple task fMRI modalities reveals a positive to negative axis across participants. Organization for Human Brain Mapping, Singapore, June 18-21 2018. \\\

Wheeler AL, Trossman R, Stojanovski S, Jacobs G, Stefanik L, \textcolor{highlight}{Viviano J}, Voineskos AN. Neurobiologically derived clusters differentiate youth based on internalizing symptom load. \textit{American College of Neuropsychopharmacology}. \\\

Overton D, Bhagwat N, \textcolor{highlight}{Viviano JD}, Voineskos AN. Identifying a psychosis spectrum disorder group using arterial spin-labeled fMRI and support vector machines. \textit{Society for Biological Psychiatry}. \\\

Bhagwat N, Patel S, \textcolor{highlight}{Viviano JD}, Voineskos AN, Chakravarty MM, and Alzheimer's Disease Neuroimaging Initiative. Modeling and prediciton of clinical symptom trajectories in Alzheimer's disease using longitudinal data. \textit{Human Brain Mapping}. \\\

Trossman R, Stojanovski S, \textcolor{highlight}{Viviano J}, Voineskos A, Wheeler A. Internalizing Symptoms are Differentially Associated with Resting State Default Mode Connectivity in Youth with a History of Traumatic Brain Injury [Abstract]. \textit{Society for Biological Psychiatry, 81(10), S245-S246}. \\\

Stojanovski S, Felsky D, \textcolor{highlight}{Viviano JD}, Shahab S, Bangali R, Burton C, O'Donnell LJ, Szatmari P, Chakravarty MM, Ameis S, Schachar R, Voineskos AN, Wheeler Al. ADHD sumptoms in youth are differentially associated with polygentic risk and neural substrates if there is a history of traumatic brain injury [Abstract]. \textit{Society for Biological Psychiatry, 81(10), S103}. \\\

Shahab S, Dickie E, \textcolor{highlight}{Viviano JD}, Foussias G, Voineskos A. Altered Functional Organization in Schizophrenia. \textit{Society for Biological Psychiatry, 81(10), S392}. \\\

Hawco C, Voineskos AN, Steeves JKE, Dickie EW, \textcolor{highlight}{Viviano JD}, Daskalakis ZJ. Spread of Activity Following TMS is correlated with Intrinsic Resting Connectivity with the Target Region: A concurrent TMS-fMRI study [Abstract]. \textit{Society for Biological Psychiatry, 81(10), S383}. \\\

Hawco C, Voineskos AN, Steeves JKE, Dickie EW, \textcolor{highlight}{Viviano JD}, Daskalakis ZJ. Spread of Activity Following TMS is correlated with Intrinsic Resting Connectivity with the Target Region: A concurrent TMS-fMRI study. \textit{Cognitive Neuroscience Society}. \\\

Ameis S, Lerch J, Taylor M, Lee W, \textcolor{highlight}{Viviano J}, Pipitone J, Nazeri A, Croarkin P, Brian J, Crosbie J, Voineskos A, Soreni N, Schachar R, Szatmari P, Arnold P, Anagnostou E. 2015. Common and distinct white matter markers in children with attention-deficit/hyperactivity disorder, obsessive compulsive disorder and autism spectrum disorder [Abstract]. \textit{Neuropsychopharmacology, 40, S292-S293}. \\\

Wheeler A, Felsky D, \textcolor{highlight}{Viviano J}, Nazeri A, Lerch J, Chakravarty M, Voineskos A. 2015. The brain-derived neurotrophic factor val66met polymorphism is associated with altered amygdala-cortical structural covariance in adolescence [Abstract]. \textit{Neuropsychopharmacology, 40, S288-S289}. \\\

\textcolor{highlight}{Viviano JD}, Park MTM, Voineskos AN, Chakravarty MM. 2015. Structural and functional connectivity of the human cerebellum using anatomical cerebellar priors. \textit{Human Brain Mapping}. \\\

Stevenson RA, Brown-Lavoie S, Segers M, Bebko J, Stevens WD, \textcolor{highlight}{Viviano J}, Baum S, Ferber S, Barense MD, Wallace MT. 2014. Impaired neural processing efficiency of multisensory integration in autism spectrum disorders. \textit{Society for Neuroscience}. \\\

\textcolor{highlight}{Viviano JD}, Stevens WD, Schneider KA. 2014. A method for probing the resonance of neural populations using fMRI. \textit{Lake Ontario Visionary Establishment}. \\\

McKetton L, \textcolor{highlight}{Viviano J}, Schneider KA. 2013. Resolving the individual layers of the human lateral geniculate nucleus using high-resolution structural MRI [Abstract]. \textit{Journal of Vision, 13(9): 554}. \\\

Chouinard PA, McLean AA, Sperandio I, \textcolor{highlight}{Viviano J}, Schneider KA, Goodale MA. 2013. Magnocellular and parvocellular fMRI activation in separate subdivisions of the human lateral geniculate nucleus. \textit{Canadian Neuroscience Meeting}. \\\

\textcolor{highlight}{Viviano JD}, Schneider KA. 2013. Tremotopic mapping of the human thalamic reticular nucleus. \textit{York University Centre for Vision Research Conference}. \\\

DeSimone K, \textcolor{highlight}{Viviano J}, Schneider KA. 2013. Population receptive field estimation in the human lateral geniculate nucleus. \textit{Human Brain Mapping}. \\\

\textcolor{highlight}{Viviano JD}, Schneider KA. 2012. Flicker modulation isolates the magnocellular layers of the human lateral geniculate nucleus. \textit{Society for Neuroscience}. \\\

\textcolor{highlight}{Viviano JD}, DeSimone K, Schneider KA. 2012. Intrinsic functional connectivity of the human lateral geniculate nucleus [Abstract]. \textit{Journal of Vision, 12(9): 382}. \\\

%%%%%%%%%%%%%%%%%%%%%%%%%%%%%%%%%%%%%%%%%%%%%%%%%%%%%%%%%%%%%%%%%%%%%%%%%%%%%%%%
\sectionheading{TEACHING} \\\

\entry
{2019}
{Instructor}
{MAIN Workshops 2019: Deep Learning for Neuroscientists}
{A 1.5-hour course on the fundamentals of deep learning for a non-CS audiance.}

\entry
{2019}
{Instructor}
{Brainhack 2019: Deep Learning for Neuroscientists}
{Developed a 3-hour course on the fundamentals of deep learning for a non-CS audiance.}

\entry
{2015--2016}
{Instructor}
{Python for MRI Data Analysis}
{Designed and taught course on MRI data analysis using Numpy.}

\entry
{2014}
{Tutorial Leader}
{Genetics, BIOL 2040}
{Taught intro-level genetics and marked assignments.}

\entry
{2013}
{Mark Compiler}
{Introduction to Biology, BIOL 1000}
{Automated this position using Python.}

\entry
{2012--2013}
{Tutorial Leader}
{Mind and Brain, NATS 1860}
{Led discussions about cognitive science, psychology, and neuroscience with non-science majors.}

\entry
{2012--2014}
{Tutorial Leader}
{Biology of Sex, NATS 1660}
{Taught basic genetics and cell biology to non-science majors.}

\entry
{2011--2012}
{Tutorial Leader}
{The Living Body, NATS 1610}
{Taught basic physiology to non-science majors.}

\entry
{2011}
{Tutorial Leader}
{Introduction to Biology, BIOL 1000}
{Taught intro-level biology, led labs, and marked full-length reports.}

\entry
{2011}
{Tutorial Leader}
{Statistics for Biologists, BIOL 2060}
{Taught intro-level statistics and marked assignments.}

%%%%%%%%%%%%%%%%%%%%%%%%%%%%%%%%%%%%%%%%%%%%%%%%%%%%%%%%%%%%%%%%%%%%%%%%%%%%%%%%
\sectionheading{CERTIFICATION} \\\

\entry
{2016}
{Certified System Administrator}
{Linux Foundation}
{Understanding of basic Linux environment, including system signals, networking, user access, automation, virtualisation, \& maintenance.}

%%%%%%%%%%%%%%%%%%%%%%%%%%%%%%%%%%%%%%%%%%%%%%%%%%%%%%%%%%%%%%%%%%%%%%%%%%%%%%%%
\sectionheading{INDEPENDENT COURSEWORK} \\\

\entry
{2018}
{Deep Learning Specialization}
{deeplearning.ai, Coursera}
{Hyperparameter tuning, regularization, optimization, convolutional networks, sequence models.}

\entry
{2017}
{Hadoop Platform and Application Framework}
{Coursera}
{Hadoop, mapreduce, \& Spark.}

\entry
{2017}
{Neural Networks for Machine Learning}
{Coursera}
{Overview of standard deep learning methods.}

\entry
{2016}
{Machine Learning}
{Coursera}
{Implementation of commonly-used machine learning models.}

\entry
{2016}
{The Essentials of System Administration}
{The Linux Foundation}
{Design, management, and administration of Linux infrastructure.}

\entry
{2014--2016}
{Intro to High Performance Computing}
{SciNet, University of Toronto}
{OpenMP, MPI, CUDA, Parallel Python, R.}

\entry
{2013}
{Intro to Data Science}
{Coursera}
{Python, SQL, Twitter scraping, Kaggle, Tableau.}

%%%%%%%%%%%%%%%%%%%%%%%%%%%%%%%%%%%%%%%%%%%%%%%%%%%%%%%%%%%%%%%%%%%%%%%%%%%%%%%%
\end{flushleft}

\vfill

\begin{center}
\small{Compiled on \usdate\today.}
\end{center}

\end{document}
